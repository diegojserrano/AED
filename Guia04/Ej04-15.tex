Realice un programa que le permita calcular el Índice de Masa Corporal (IMC) de una persona en función de su peso (en Kg.) y su altura (en mts.), sabiendo que el IMC es igual al peso dividido la altura al cuadrado. En función del valor del IMC, el programa debe mostrar por pantalla el diagnóstico resultante del análisis del índice según las siguientes situaciones:

\begin{itemize}
	\item Si el IMC es menor o igual a 16: ``Necesita asistencia de un médico, los riesgos para su salud son muy altos''.
	\item Si el IMC es menor o igual a 17: ``Usted tiene infrapeso, aliméntese más''.
	\item Si el IMC es menor o igual a 18: ``Usted tiene bajo peso, aliméntese mejor''.
	\item Si el IMC es mayor a 18 y menor o igual a 26: ``Usted tiene un peso saludable, continúe así!''.
	\item Si el IMC es mayor a 26 y menor a 30: ``Tiene sobrepeso de grado I, hoy es un buen día para empezar a hacer ejercicios''.
	\item Si el IMC es mayor o igual a 30 y menor o igual a 35: ``Tiene obesidad de grado II, necesita el apoyo de un plan nutricional''.
	\item Si el IMC es mayor a 35 y menor o igual a 40: ``Tiene obesidad grado III (pre-mórbida), consulte con su médico los riesgos para su salud''.
	\item Si el IMC es mayor a 40: ``Usted tiene obesidad de grado IV (mórbida), los riesgos para su salud son muy altos, consulte con su médico a la brevedad''.
\end{itemize}