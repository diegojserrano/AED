Un comerciante tiene a la venta 3 tipos de artículos principales. Conociendo la cantidad vendida de cada artículo y el precio unitario de cada artículo, hacer un programa que determine cuál fue el producto que realizó el mayor aporte en los ingresos y el porcentaje que dicho aporte significa en el ingreso absoluto de los 3 artículos sumados. 

Ese porcentaje se calcula así:

\begin{eqnarray*}
\textrm{Absoluto} & = & 100\%\\
\textrm{Mayor aporte} & = & x\% \\
\end{eqnarray*}

Por lo tanto:    

\[
x = \frac{\textrm{Mayor aporte} \cdot 100}{ \textrm{absoluto}}
\]

