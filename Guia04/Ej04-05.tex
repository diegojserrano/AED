Se tienen los datos de tres postulantes a un empleo, a los que se les realizó un test de capacitación. Por cada postulante, se tiene entonces la siguiente información:  nombre del postulante, cantidad total de preguntas que se le realizaron y cantidad de preguntas que contestó correctamente. 

Se pide confeccionar un programa que lea los datos de los tres postulantes, informe el nivel de cada uno según los criterios de aprobación que se indican mas abajo, e indique finalmente el nombre del postulante que ganó el puesto. Los criterios de aprobación son los siguientes, en función del porcentaje de respuestas correctas sobre el total de preguntas realizadas a cada postulante:

\begin{itemize}
\item Nivel Superior:       Porcentaje \geqslant\ 90\%
\item Nivel Medio:          75\% \leqslant\ Porcentaje <\ 90\%
\item Nivel Regular:        50\% \leqslant\ Porcentaje <\ 75\%
\item Fuera de Nivel:       Porcentaje <\ 50\%
\end{itemize}