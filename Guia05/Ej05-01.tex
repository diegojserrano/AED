Según la Ley Electoral de la República Argentina, el Presidente y el Vicepresidente se eligen de acuerdo a las siguientes reglas:

Artículo 149. — Resultará electa la fórmula que obtenga más del cuarenta y cinco por ciento (45\%) de los votos afirmativos válidamente emitidos; en su defecto, aquella que hubiere obtenido el cuarenta por ciento (40\%) por lo menos de los votos afirmativos válidamente emitidos y, además, existiere una diferencia mayor de diez puntos porcentuales respecto del total de los votos afirmativos válidamente emitidos, sobre la fórmula que le sigue en número de votos.

Artículo 150. — Si ninguna fórmula alcanzare esas mayorías y diferencias de acuerdo al escrutinio ejecutado por las Juntas Electorales, y cuyo resultado único para toda la Nación será anunciado por la Asamblea Legislativa atento lo dispuesto por el artículo 120 de la presente ley, se realizará una segunda vuelta dentro de los treinta (30) días.

Artículo 151. — En la segunda vuelta participarán solamente las dos fórmulas más votadas en la primera, resultando electa la que obtenga mayor número de votos afirmativos válidamente emitidos.

Desarrollar un programa que permita ingresar, para los 3 partidos más votados: fórmula (presidente + vice) y cantidad de votos obtenidos.

Luego determinar:

\begin{itemize}
	\item Qué fórmula obtuvo el mayor porcentaje.
	\item Si la fórmula resulta elegida o se requiere segunda vuelta. En este caso, indicar también quienes participan de la segunda vuelta
\end{itemize}
