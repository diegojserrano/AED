Son muchas las concesionarias que en el último tiempo se han visto afectados por el poderoso virus PANTEVIL que ataca los registros de automóviles, alterando el identificador de la patente de los mismos. Sabemos que el identificador de patente de un automóvil está compuesto por 3 letras en mayúsculas y 3 números, por ejemplo AED335.

Luego de un exhaustivo análisis de los registros infectados, se logró decodificar cómo funciona el virus. Vamos a separar en letras y números para explicar cómo codifica el virus:

Letras:
\begin{itemize}
	\item El virus transforma cada carácter en el entero Unicode al que representa.
	\item Luego chequea si esos tres números son iguales, en caso afirmativo reemplaza los valores del primer y último número por valores aleatorios generados entre 65 y 90.
	\item Una vez que tiene esos números, los convierte a cadena de caracteres y los concatena, anteponiendo:
	\begin{itemize}
		\item Un signo @ en caso que los tres números hayan sido iguales o
		\item Un signo \& en caso contrario.	
	\end{itemize}
	
\end{itemize}
Números:

\begin{itemize}
	\item Para codificar los números el virus utiliza una cadena con 5 caracteres
	\item El primer carácter codifica:
	\begin{itemize}
		\item Un signo + si los 3 números eran pares y les suma 1 a cada número.
		\item Un signo - si los 3 números no son pares.
	\end{itemize}
	\item En el segundo carácter el virus pone:
	\begin{itemize}
		\item Un signo \# en caso que el primer número y el segundo son iguales, y cambia el valor de segundo numero por el tercero.
		\item Un signo \$ en caso que el primer número y el tercero son iguales, y cambia el valor de tercer numero por el segundo.
		\item Un signo * en caso que el segundo número y el tercero son iguales, y cambia el valor de tercer numero por el primero.
		\item Un signo ! en caso que los 3 números sean diferentes.
		\item En el tercero, cuarto y quinto carácter el virus simplemente concatena los números resultantes.
	\end{itemize}
\end{itemize}

Finalmente y como si fuera poco, una vez codificados, el virus invierte el orden de los números y letras. Esto es primero coloca los números codificados y luego las letras codificadas. Veamos algunos ejemplos:

{\renewcommand{\arraystretch}{1.5}

\begin{center}	
\begin{tabular}{c | c}
	Patente sin virus & Patente con virus \\ 
	\hline
    AED335 & -\#355\&656968 \\
    PEP456 & -!456\&806980 \\
    RRR682 & +!793@898280 \\
\end{tabular} 
\end{center}
}

Debido a la gravedad del caso, las concesionarias afectadas no pueden realizar ninguna venta hasta que no se reparen los archivos dañados, es que nos han solicitado con urgencia un reparador para el virus PANTEVIL.